\documentclass{article}
\usepackage{geometry}
\usepackage[utf8]{inputenc}
\usepackage[T1]{fontenc}
\usepackage{graphicx}
\usepackage{lmodern}
\usepackage{microtype}
\usepackage{amsmath}
\usepackage{amssymb,latexsym}
\usepackage{hyperref}

\usepackage[estonian]{babel}
\usepackage{enumerate}

\title{Graafiteooria} \author{Kaur Aare Saar \\
  \href{mailto:kauraare@gmail.com}{kauraare@gmail.com}}

\begin{document}
\maketitle

\section{Mõisted}
\begin{itemize}
\item \emph{Graaf} $G=(V,E)$ koosneb \emph{tippudest} $V$ ja
  \emph{servadest} $E$.
\item Graafi tipud on \emph{ühendatud}, kui need on graafi ühe serva
  otspunktideks.
\item Graaf on \emph{suunatud}, kui graafi serva otspunktid on
  järjestatud.
\item Omavahel ühendatud tippude jada nimetatakse \emph{teeks}.
\item Kui iga kahe tipu vahel leidub tee, siis graaf on \emph{sidus}.
\item \emph{Tsükkel} on tee, mille algus ja lõpppunkt kattuvad ning ei
  sisalda kattuvaid servasid.
\item \emph{Euleri tsükkel} on tsükkel, mis sisaldab kõiki graafi
  servasid ja tippe.
\item \emph{Hamiltoni tsükkel} on tsükkel, mis läbib kõiki graafi
  tippe täpselt ühe korra.
\item \emph{Puu} on sidus graaf, mis ei sisalda tsüklit.
\item Tipu \emph{aste} $d$ on sellest lähtuvate servade arv.
\item Graafi $G$ \emph{täiendgraaf} $\overline{G}$ on graaf, mis
  sisaldab ainult neid servasid, mis puuduvad graafis $G$.
\end{itemize}

\section{Soojendus}
\begin{enumerate}
\item Tõesta, et $n$-tipulises graafis ($n \geq 2$) leidub kaks tippu,
  mille aste on võrdne.
\item Leia servade arv $n$-tipulises puus.
\item Tõesta, et vähemalt üks graafidest $G$ ja $\overline{G}$ on
  sidus.
\item $n$-tipulise graafi $G$ mis tahes kahe tipu astmete summa on
  suurem kui $n$. Tõesta, et $G$ sisaldab Hamiltoni tsüklit.
\item Tõesta, et sidus graaf sisaldab Euleri tsüklit parajasti siis,
  kui kõikide tippude aste on paarisarv.
\item \emph{(Diraci teoreem).} Tõesta, et $n$ tipuga graaf sisaldab
  Hamiltoni tsüklit, kui iga tipu aste on vähemalt $n/2$.
\end{enumerate}

\section{Ülesanded}
\begin{enumerate}
\item Club Olegis sai kokku $n$ inimest. Millise vähima $n$ korral,
  leiduvad nende hulgas kindlasti kolm sellist, kes kõik üksteist
  tunnevad, või kolm sellist, kellest ükski paar teineteist ei tunne?
\item Toas on $2n$ inimest, kellest igaühel on ülimalt $n-1$
  vaenlast. Tõesta, et need $2n$ inimset saavad istuda ümber
  ringikujulise laua nii, et vaenlased ei istuks kõrvuti.
\item \emph{(Euleri teoreem).} Kumeras hulktahukas on $E$ serva, $F$
  tahku ja $V$ tippu. Tõesta, et $E+2=F+V$.
\item \emph{(USAMO 1999).}  Mõõtmetega $n \times n$ ruudustikus
  värvitakse mõned ruudud mustaks ja kõik ülejäänud valgeks nii, et on
  täidetud järgised tingimused:
  \begin{enumerate}[(i)]
  \item Igal valgel ruudul on ühine külg mingi musta ruuduga
  \item Mistahes kahe musta ruudu korral leidub selline järjend
    mustadest ruutudest, mille esimeseks ja viimaseks ruuduks on need
    kaks ruutu ning mille igal kahel järjestikusel ruudul on ühine
    külg.
  \end{enumerate}
  Tõesta, et ruudustikus on vähemalt $\frac{n^2-2}{3}$ musta ruutu.
\item \emph{(Balti tee 1997).} Metsas elavad $n$ looma ($n \geq 3$),
  igaüks oma urus ning iga kaht urgu ühendab täpselt üks
  rada. Metsakuninga valimise eel teevad mõned loomad
  valimiskampaaniat. Iga kampaaniat tegev loom külastab iga teise
  looma urgu täpselt korra, liigub urust urgu ainult mööda radu, ei
  pööra urgude vahel kusagil ühelt rajalt teisele ning pöördub lõpuks
  tagasi oma urgu. Samuti on teada, et iga rada kasutab ülimalt üks
  kampaaniat tegev loom.
  \begin{enumerate}[(i)]
  \item Tõesta, et iga algarvulise $n$ korral on kampaaniat tegevate
    loomade suurim võimalik arv $\frac{n-1}{2}$.
  \item Leia kampaaniat tegevate loomade suurim võimalik arv, kui
    $n=9$.
  \end{enumerate}
\item \emph{(BT 1991).} Lossis on teatud arv saale ja $n$ ust. Iga uks
  ühendab mingit kaht saali või viib lossist välja. Igal saalil on
  vähemalt 2 ust. Rüütel siseneb lossi ja edaspidi võib ta igast
  saalist väljuda mistahes ukse kaudu peale selle, mille kaudu ta
  viimati sellesse saali sisenes.  Leia strateegia, mille abil rüütel
  jõuab lossist välja, olles läbinud mitte rohkem kui $2n$
  saali. (Saali läbimine läheb arvesse iga kord, kui rüütel sellesse
  siseneb.)
\item \emph{(IMO 2001 Shortlist).} Olgu $k$-klikk hulk $k$ inimesega
  nii, et selle iga paar tunneb teineteist. Club Olegis, on igas
  $3$-kliki paaris vähemalt üks kattuv inimene, aga pole ühtegi
  $5$-klikki. Tõesta, et Club Olegis on ülimalt kaks inimest, kelle
  lahkumise korral ei jääks Club Olegi mitte ühtegi $3$-klikki.
\item \emph{(Balti Tee 1994).} Kuningas otsustas lasta ehitada oma
  kuningriigi 13 asustamata saarele kokku 25 linna ning panna käima
  praamiliini iga erinevatel saartel olevate linnade paari
  vahel. Kuidas peaks kuningas linnad saarte vahel jaotama, et igaühel
  $13$ saarest oleks vähemalt üks linn ja praamiliinide koguarv oleks
  vähim?
\item \emph{(Canada 2006).} Turniiril, kus osaleb $2n+1$ võistkonda,
  mängivad kõik võistkondade paarid täpselt ühe korra. Kolm võistkonda
  moodustavad \emph{surnud ringi}, kui $X$ võidab $Y$, $Y$ võidab $Z$
  ja $Z$ võidab $X$.
  \begin{enumerate}[(i)]
  \item Leia minimaalne surnud ringide arv.
  \item Leia maksimaalne surnud ringide arv.
  \end{enumerate}
\end{enumerate}

\end{document}
